\documentclass{scrarticle}

% Hier kommt die Präambel hin

% erst werden die Pakete definiert
\usepackage[ngerman]{babel} % neue Deutsche Rechtschreibung
\usepackage[utf8]{inputenc} % utf8-encoding
\usepackage[T1]{fontenc}    % westeuropäisches font-encoding für schonere fonts
\usepackage[onehalfspacing]{setspace}  % Zeilenabstand auf 1.5 setzen
\usepackage[hmargin={2.5cm,2.5cm}]{geometry} % Seitenränder wie im Merkblatt vorgeschrieben auf 2,5cm setzen


\usepackage{graphicx} % Paket für Einbindung von Bildern

\usepackage[german=guillemets]{csquotes} % sehr schöne Anführungszeichen

%disable before printing -> pressable links will show at the printed page!!! Only used for digital pdf's
\usepackage[breaklinks=true]{hyperref}

% Optionen fürs Layout
\KOMAoptions{
	parskip=full, % kein Abstand zwischen Absätze, nur Einrückungen
	fontsize=12pt,
	paper=a4,
	pagesize=auto,
	bibliography=nottotoc, % Literaturverzeichnis wird nicht ins Inhaltsverzeichnis aufgenommen
	bibliography=openstyle, % Art des Literaturverzeichnis
}

% Überschriften mit Serifen
\addtokomafont{section}{\rmfamily}
\addtokomafont{subsection}{\rmfamily}
\addtokomafont{subsubsection}{\rmfamily}

\usepackage[style=reading]{biblatex}
\bibliography{bib/Seminararbeit}

\begin{document}
    \section{Einleitung}
    Bei der Betrachtung der Benennung der Epochen in der menschlichen Geschichte ist bemerkbar, dass hier mehrmals die bedeutendste technische Errungenschaft aus diesem Zeitraum zur Namensgebung verwendet wird. Das ist natürlich sinnvoll da diese Entwicklungen im großen Ausmaß das Leben der Menschen beeinflussten und auch für den weiteren Verlauf der Geschichte ausschlaggebend sind. Diese Namen sind  wie \enquote{Bronzezeit} recht selbsterklärend, hier wurde die Metallverarbeitung, vor allem mit Bronze erfunden und revolutionierte die Waffentechnik und ermöglichte auch viele andere neuartige Gegenstände.\autocite{BronzezeitEuropaDeutschland} Schwieriger wird es dann schon bei dem Titel unseres Zeitabschnittes: \enquote{Digitales Informationszeitalter}


\end{document}